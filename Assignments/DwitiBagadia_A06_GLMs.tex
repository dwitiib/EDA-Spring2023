% Options for packages loaded elsewhere
\PassOptionsToPackage{unicode}{hyperref}
\PassOptionsToPackage{hyphens}{url}
%
\documentclass[
]{article}
\usepackage{amsmath,amssymb}
\usepackage{lmodern}
\usepackage{iftex}
\ifPDFTeX
  \usepackage[T1]{fontenc}
  \usepackage[utf8]{inputenc}
  \usepackage{textcomp} % provide euro and other symbols
\else % if luatex or xetex
  \usepackage{unicode-math}
  \defaultfontfeatures{Scale=MatchLowercase}
  \defaultfontfeatures[\rmfamily]{Ligatures=TeX,Scale=1}
\fi
% Use upquote if available, for straight quotes in verbatim environments
\IfFileExists{upquote.sty}{\usepackage{upquote}}{}
\IfFileExists{microtype.sty}{% use microtype if available
  \usepackage[]{microtype}
  \UseMicrotypeSet[protrusion]{basicmath} % disable protrusion for tt fonts
}{}
\makeatletter
\@ifundefined{KOMAClassName}{% if non-KOMA class
  \IfFileExists{parskip.sty}{%
    \usepackage{parskip}
  }{% else
    \setlength{\parindent}{0pt}
    \setlength{\parskip}{6pt plus 2pt minus 1pt}}
}{% if KOMA class
  \KOMAoptions{parskip=half}}
\makeatother
\usepackage{xcolor}
\usepackage[margin=2.54cm]{geometry}
\usepackage{color}
\usepackage{fancyvrb}
\newcommand{\VerbBar}{|}
\newcommand{\VERB}{\Verb[commandchars=\\\{\}]}
\DefineVerbatimEnvironment{Highlighting}{Verbatim}{commandchars=\\\{\}}
% Add ',fontsize=\small' for more characters per line
\usepackage{framed}
\definecolor{shadecolor}{RGB}{248,248,248}
\newenvironment{Shaded}{\begin{snugshade}}{\end{snugshade}}
\newcommand{\AlertTok}[1]{\textcolor[rgb]{0.94,0.16,0.16}{#1}}
\newcommand{\AnnotationTok}[1]{\textcolor[rgb]{0.56,0.35,0.01}{\textbf{\textit{#1}}}}
\newcommand{\AttributeTok}[1]{\textcolor[rgb]{0.77,0.63,0.00}{#1}}
\newcommand{\BaseNTok}[1]{\textcolor[rgb]{0.00,0.00,0.81}{#1}}
\newcommand{\BuiltInTok}[1]{#1}
\newcommand{\CharTok}[1]{\textcolor[rgb]{0.31,0.60,0.02}{#1}}
\newcommand{\CommentTok}[1]{\textcolor[rgb]{0.56,0.35,0.01}{\textit{#1}}}
\newcommand{\CommentVarTok}[1]{\textcolor[rgb]{0.56,0.35,0.01}{\textbf{\textit{#1}}}}
\newcommand{\ConstantTok}[1]{\textcolor[rgb]{0.00,0.00,0.00}{#1}}
\newcommand{\ControlFlowTok}[1]{\textcolor[rgb]{0.13,0.29,0.53}{\textbf{#1}}}
\newcommand{\DataTypeTok}[1]{\textcolor[rgb]{0.13,0.29,0.53}{#1}}
\newcommand{\DecValTok}[1]{\textcolor[rgb]{0.00,0.00,0.81}{#1}}
\newcommand{\DocumentationTok}[1]{\textcolor[rgb]{0.56,0.35,0.01}{\textbf{\textit{#1}}}}
\newcommand{\ErrorTok}[1]{\textcolor[rgb]{0.64,0.00,0.00}{\textbf{#1}}}
\newcommand{\ExtensionTok}[1]{#1}
\newcommand{\FloatTok}[1]{\textcolor[rgb]{0.00,0.00,0.81}{#1}}
\newcommand{\FunctionTok}[1]{\textcolor[rgb]{0.00,0.00,0.00}{#1}}
\newcommand{\ImportTok}[1]{#1}
\newcommand{\InformationTok}[1]{\textcolor[rgb]{0.56,0.35,0.01}{\textbf{\textit{#1}}}}
\newcommand{\KeywordTok}[1]{\textcolor[rgb]{0.13,0.29,0.53}{\textbf{#1}}}
\newcommand{\NormalTok}[1]{#1}
\newcommand{\OperatorTok}[1]{\textcolor[rgb]{0.81,0.36,0.00}{\textbf{#1}}}
\newcommand{\OtherTok}[1]{\textcolor[rgb]{0.56,0.35,0.01}{#1}}
\newcommand{\PreprocessorTok}[1]{\textcolor[rgb]{0.56,0.35,0.01}{\textit{#1}}}
\newcommand{\RegionMarkerTok}[1]{#1}
\newcommand{\SpecialCharTok}[1]{\textcolor[rgb]{0.00,0.00,0.00}{#1}}
\newcommand{\SpecialStringTok}[1]{\textcolor[rgb]{0.31,0.60,0.02}{#1}}
\newcommand{\StringTok}[1]{\textcolor[rgb]{0.31,0.60,0.02}{#1}}
\newcommand{\VariableTok}[1]{\textcolor[rgb]{0.00,0.00,0.00}{#1}}
\newcommand{\VerbatimStringTok}[1]{\textcolor[rgb]{0.31,0.60,0.02}{#1}}
\newcommand{\WarningTok}[1]{\textcolor[rgb]{0.56,0.35,0.01}{\textbf{\textit{#1}}}}
\usepackage{graphicx}
\makeatletter
\def\maxwidth{\ifdim\Gin@nat@width>\linewidth\linewidth\else\Gin@nat@width\fi}
\def\maxheight{\ifdim\Gin@nat@height>\textheight\textheight\else\Gin@nat@height\fi}
\makeatother
% Scale images if necessary, so that they will not overflow the page
% margins by default, and it is still possible to overwrite the defaults
% using explicit options in \includegraphics[width, height, ...]{}
\setkeys{Gin}{width=\maxwidth,height=\maxheight,keepaspectratio}
% Set default figure placement to htbp
\makeatletter
\def\fps@figure{htbp}
\makeatother
\setlength{\emergencystretch}{3em} % prevent overfull lines
\providecommand{\tightlist}{%
  \setlength{\itemsep}{0pt}\setlength{\parskip}{0pt}}
\setcounter{secnumdepth}{-\maxdimen} % remove section numbering
\ifLuaTeX
  \usepackage{selnolig}  % disable illegal ligatures
\fi
\IfFileExists{bookmark.sty}{\usepackage{bookmark}}{\usepackage{hyperref}}
\IfFileExists{xurl.sty}{\usepackage{xurl}}{} % add URL line breaks if available
\urlstyle{same} % disable monospaced font for URLs
\hypersetup{
  pdftitle={Assignment 6: GLMs (Linear Regressios, ANOVA, \& t-tests)},
  pdfauthor={Dwiti Bagadia},
  hidelinks,
  pdfcreator={LaTeX via pandoc}}

\title{Assignment 6: GLMs (Linear Regressios, ANOVA, \& t-tests)}
\author{Dwiti Bagadia}
\date{Spring 2023}

\begin{document}
\maketitle

\hypertarget{overview}{%
\subsection{OVERVIEW}\label{overview}}

This exercise accompanies the lessons in Environmental Data Analytics on
generalized linear models.

\hypertarget{directions}{%
\subsection{Directions}\label{directions}}

\begin{enumerate}
\def\labelenumi{\arabic{enumi}.}
\tightlist
\item
  Rename this file
  \texttt{\textless{}FirstLast\textgreater{}\_A06\_GLMs.Rmd} (replacing
  \texttt{\textless{}FirstLast\textgreater{}} with your first and last
  name).
\item
  Change ``Student Name'' on line 3 (above) with your name.
\item
  Work through the steps, \textbf{creating code and output} that fulfill
  each instruction.
\item
  Be sure to \textbf{answer the questions} in this assignment document.
\item
  When you have completed the assignment, \textbf{Knit} the text and
  code into a single PDF file.
\end{enumerate}

\hypertarget{set-up-your-session}{%
\subsection{Set up your session}\label{set-up-your-session}}

\begin{enumerate}
\def\labelenumi{\arabic{enumi}.}
\item
  Set up your session. Check your working directory. Load the tidyverse,
  agricolae and other needed packages. Import the \emph{raw} NTL-LTER
  raw data file for chemistry/physics
  (\texttt{NTL-LTER\_Lake\_ChemistryPhysics\_Raw.csv}). Set date columns
  to date objects.
\item
  Build a ggplot theme and set it as your default theme.
\end{enumerate}

\begin{Shaded}
\begin{Highlighting}[]
\CommentTok{\#1 }

\FunctionTok{library}\NormalTok{(tidyverse)}
\end{Highlighting}
\end{Shaded}

\begin{verbatim}
## Warning: package 'tidyverse' was built under R version 4.2.2
\end{verbatim}

\begin{verbatim}
## Warning: package 'ggplot2' was built under R version 4.2.2
\end{verbatim}

\begin{verbatim}
## Warning: package 'tibble' was built under R version 4.2.2
\end{verbatim}

\begin{verbatim}
## Warning: package 'tidyr' was built under R version 4.2.2
\end{verbatim}

\begin{verbatim}
## Warning: package 'readr' was built under R version 4.2.2
\end{verbatim}

\begin{verbatim}
## Warning: package 'purrr' was built under R version 4.2.2
\end{verbatim}

\begin{verbatim}
## Warning: package 'dplyr' was built under R version 4.2.2
\end{verbatim}

\begin{verbatim}
## Warning: package 'stringr' was built under R version 4.2.2
\end{verbatim}

\begin{verbatim}
## Warning: package 'forcats' was built under R version 4.2.2
\end{verbatim}

\begin{verbatim}
## Warning: package 'lubridate' was built under R version 4.2.2
\end{verbatim}

\begin{verbatim}
## -- Attaching core tidyverse packages ------------------------ tidyverse 2.0.0 --
## v dplyr     1.1.0     v readr     2.1.4
## v forcats   1.0.0     v stringr   1.5.0
## v ggplot2   3.4.1     v tibble    3.1.8
## v lubridate 1.9.2     v tidyr     1.3.0
## v purrr     1.0.1     
## -- Conflicts ------------------------------------------ tidyverse_conflicts() --
## x dplyr::filter() masks stats::filter()
## x dplyr::lag()    masks stats::lag()
## i Use the ]8;;http://conflicted.r-lib.org/conflicted package]8;; to force all conflicts to become errors
\end{verbatim}

\begin{Shaded}
\begin{Highlighting}[]
\FunctionTok{library}\NormalTok{(agricolae)}
\end{Highlighting}
\end{Shaded}

\begin{verbatim}
## Warning: package 'agricolae' was built under R version 4.2.2
\end{verbatim}

\begin{Shaded}
\begin{Highlighting}[]
\FunctionTok{library}\NormalTok{(here)}
\end{Highlighting}
\end{Shaded}

\begin{verbatim}
## Warning: package 'here' was built under R version 4.2.2
\end{verbatim}

\begin{verbatim}
## here() starts at C:/Users/dwiti/OneDrive - University of North Carolina at Chapel Hill/EDA/EDA-Spring2023
\end{verbatim}

\begin{Shaded}
\begin{Highlighting}[]
\FunctionTok{library}\NormalTok{(cowplot)}
\end{Highlighting}
\end{Shaded}

\begin{verbatim}
## Warning: package 'cowplot' was built under R version 4.2.2
\end{verbatim}

\begin{verbatim}
## 
## Attaching package: 'cowplot'
## 
## The following object is masked from 'package:lubridate':
## 
##     stamp
\end{verbatim}

\begin{Shaded}
\begin{Highlighting}[]
\FunctionTok{library}\NormalTok{(ggplot2)}
\FunctionTok{library}\NormalTok{(lubridate)}

\FunctionTok{getwd}\NormalTok{()}
\end{Highlighting}
\end{Shaded}

\begin{verbatim}
## [1] "C:/Users/dwiti/OneDrive - University of North Carolina at Chapel Hill/EDA/EDA-Spring2023"
\end{verbatim}

\begin{Shaded}
\begin{Highlighting}[]
\NormalTok{NTL\_LTER }\OtherTok{\textless{}{-}} \FunctionTok{read.csv}\NormalTok{(}\StringTok{"./Data/Raw/NTL{-}LTER\_Lake\_ChemistryPhysics\_Raw.csv"}\NormalTok{, }\AttributeTok{stringsAsFactors =} \ConstantTok{TRUE}\NormalTok{)}

\FunctionTok{class}\NormalTok{(NTL\_LTER}\SpecialCharTok{$}\NormalTok{sampledate)}
\end{Highlighting}
\end{Shaded}

\begin{verbatim}
## [1] "factor"
\end{verbatim}

\begin{Shaded}
\begin{Highlighting}[]
\NormalTok{NTL\_LTER}\SpecialCharTok{$}\NormalTok{sampledate }\OtherTok{=} \FunctionTok{ymd}\NormalTok{(NTL\_LTER}\SpecialCharTok{$}\NormalTok{sampledate)}
\end{Highlighting}
\end{Shaded}

\begin{verbatim}
## Warning: 33138 failed to parse.
\end{verbatim}

\begin{Shaded}
\begin{Highlighting}[]
\CommentTok{\#2 creating theme }
\FunctionTok{library}\NormalTok{(ggthemes)}
\end{Highlighting}
\end{Shaded}

\begin{verbatim}
## Warning: package 'ggthemes' was built under R version 4.2.2
\end{verbatim}

\begin{verbatim}
## 
## Attaching package: 'ggthemes'
## 
## The following object is masked from 'package:cowplot':
## 
##     theme_map
\end{verbatim}

\begin{Shaded}
\begin{Highlighting}[]
\NormalTok{tweetheme }\OtherTok{\textless{}{-}} \FunctionTok{theme\_classic}\NormalTok{(}\AttributeTok{base\_size =} \DecValTok{14}\NormalTok{) }\SpecialCharTok{+} 
  \FunctionTok{theme}\NormalTok{(}\AttributeTok{axis.text =} \FunctionTok{element\_text}\NormalTok{(}\AttributeTok{color =} \StringTok{"darkgrey"}\NormalTok{),}
        \AttributeTok{axis.ticks =} \FunctionTok{element\_line}\NormalTok{(}\AttributeTok{color =} \StringTok{"darkgrey"}\NormalTok{),}
        \AttributeTok{plot.background =} \FunctionTok{element\_rect}\NormalTok{(}\AttributeTok{color =} \StringTok{"white"}\NormalTok{),}
        \AttributeTok{legend.position =} \StringTok{"top"}\NormalTok{)}
\end{Highlighting}
\end{Shaded}

\hypertarget{simple-regression}{%
\subsection{Simple regression}\label{simple-regression}}

Our first research question is: Does mean lake temperature recorded
during July change with depth across all lakes?

\begin{enumerate}
\def\labelenumi{\arabic{enumi}.}
\setcounter{enumi}{2}
\item
  State the null and alternative hypotheses for this question:
  \textgreater{} Answer: H0: The mean lake temperature recorded during
  July doesn't change with depth across all lakes. Ha: The mean lake
  temperature recorded during July changes with depth across all lakes.
\item
  Wrangle your NTL-LTER dataset with a pipe function so that the records
  meet the following criteria:
\end{enumerate}

\begin{itemize}
\tightlist
\item
  Only dates in July.
\item
  Only the columns: \texttt{lakename}, \texttt{year4}, \texttt{daynum},
  \texttt{depth}, \texttt{temperature\_C}
\item
  Only complete cases (i.e., remove NAs)
\end{itemize}

\begin{enumerate}
\def\labelenumi{\arabic{enumi}.}
\setcounter{enumi}{4}
\tightlist
\item
  Visualize the relationship among the two continuous variables with a
  scatter plot of temperature by depth. Add a smoothed line showing the
  linear model, and limit temperature values from 0 to 35 °C. Make this
  plot look pretty and easy to read.
\end{enumerate}

\begin{Shaded}
\begin{Highlighting}[]
\CommentTok{\#4 wrangling dataset}
\NormalTok{NTL\_LTER.selected }\OtherTok{\textless{}{-}}\NormalTok{ NTL\_LTER }\SpecialCharTok{\%\textgreater{}\%} 
  \FunctionTok{filter}\NormalTok{(}\FunctionTok{month}\NormalTok{(sampledate) }\SpecialCharTok{==} \DecValTok{7}\NormalTok{) }\SpecialCharTok{\%\textgreater{}\%} 
  \FunctionTok{select}\NormalTok{(}\StringTok{"lakename"}\NormalTok{, }\StringTok{"year4"}\NormalTok{, }\StringTok{"daynum"}\NormalTok{, }\StringTok{"depth"}\NormalTok{, }\StringTok{"temperature\_C"}\NormalTok{) }\SpecialCharTok{\%\textgreater{}\%} 
  \FunctionTok{drop\_na}\NormalTok{()}

\CommentTok{\#5 visualizing with scatter plot }
\NormalTok{temperature.depthplot }\OtherTok{\textless{}{-}} \FunctionTok{ggplot}\NormalTok{(NTL\_LTER.selected) }\SpecialCharTok{+} 
  \FunctionTok{geom\_point}\NormalTok{(}\FunctionTok{aes}\NormalTok{(}\AttributeTok{x =}\NormalTok{ depth, }\AttributeTok{y =}\NormalTok{ temperature\_C)) }\SpecialCharTok{+} 
  \FunctionTok{geom\_smooth}\NormalTok{(}\FunctionTok{aes}\NormalTok{(}\AttributeTok{x =}\NormalTok{ depth, }\AttributeTok{y =}\NormalTok{ temperature\_C), }\AttributeTok{method =} \StringTok{"lm"}\NormalTok{) }\SpecialCharTok{+} 
  \FunctionTok{ylim}\NormalTok{(}\DecValTok{0}\NormalTok{, }\DecValTok{35}\NormalTok{) }\SpecialCharTok{+} 
  \FunctionTok{labs}\NormalTok{(}\AttributeTok{title =} \StringTok{"Temperature by Depth"}\NormalTok{, }\AttributeTok{x =} \StringTok{"Depth(Mts.)"}\NormalTok{, }\AttributeTok{y =} \StringTok{"Temperature (C)"}\NormalTok{) }\SpecialCharTok{+}
\NormalTok{  tweetheme}
\FunctionTok{print}\NormalTok{(temperature.depthplot)}
\end{Highlighting}
\end{Shaded}

\begin{verbatim}
## `geom_smooth()` using formula = 'y ~ x'
\end{verbatim}

\begin{verbatim}
## Warning: Removed 6 rows containing missing values (`geom_smooth()`).
\end{verbatim}

\includegraphics{DwitiBagadia_A06_GLMs_files/figure-latex/scatterplot-1.pdf}

\begin{enumerate}
\def\labelenumi{\arabic{enumi}.}
\setcounter{enumi}{5}
\tightlist
\item
  Interpret the figure. What does it suggest with regards to the
  response of temperature to depth? Do the distribution of points
  suggest about anything about the linearity of this trend?
\end{enumerate}

\begin{quote}
Answer:
\end{quote}

\begin{enumerate}
\def\labelenumi{\arabic{enumi}.}
\setcounter{enumi}{6}
\tightlist
\item
  Perform a linear regression to test the relationship and display the
  results
\end{enumerate}

\begin{Shaded}
\begin{Highlighting}[]
\CommentTok{\#7 regression models}
\NormalTok{temperature.depth.regreesion }\OtherTok{=} \FunctionTok{lm}\NormalTok{(}\AttributeTok{data =}\NormalTok{ NTL\_LTER.selected, temperature\_C }\SpecialCharTok{\textasciitilde{}}\NormalTok{ depth)}
\FunctionTok{summary}\NormalTok{(temperature.depth.regreesion)}
\end{Highlighting}
\end{Shaded}

\begin{verbatim}
## 
## Call:
## lm(formula = temperature_C ~ depth, data = NTL_LTER.selected)
## 
## Residuals:
##     Min      1Q  Median      3Q     Max 
## -7.7641 -2.8586 -0.3779  2.6155  7.7928 
## 
## Coefficients:
##             Estimate Std. Error t value Pr(>|t|)    
## (Intercept)  21.5054     0.3261   65.95   <2e-16 ***
## depth        -1.9138     0.0567  -33.76   <2e-16 ***
## ---
## Signif. codes:  0 '***' 0.001 '**' 0.01 '*' 0.05 '.' 0.1 ' ' 1
## 
## Residual standard error: 3.65 on 392 degrees of freedom
## Multiple R-squared:  0.744,  Adjusted R-squared:  0.7434 
## F-statistic:  1139 on 1 and 392 DF,  p-value: < 2.2e-16
\end{verbatim}

\begin{Shaded}
\begin{Highlighting}[]
\FunctionTok{plot}\NormalTok{(temperature.depth.regreesion, }\DecValTok{1}\NormalTok{)}
\end{Highlighting}
\end{Shaded}

\includegraphics{DwitiBagadia_A06_GLMs_files/figure-latex/linear.regression-1.pdf}

\begin{Shaded}
\begin{Highlighting}[]
\FunctionTok{plot}\NormalTok{(temperature.depth.regreesion, }\DecValTok{2}\NormalTok{)}
\end{Highlighting}
\end{Shaded}

\includegraphics{DwitiBagadia_A06_GLMs_files/figure-latex/linear.regression-2.pdf}

\begin{Shaded}
\begin{Highlighting}[]
\FunctionTok{plot}\NormalTok{(temperature.depth.regreesion, }\DecValTok{3}\NormalTok{)}
\end{Highlighting}
\end{Shaded}

\includegraphics{DwitiBagadia_A06_GLMs_files/figure-latex/linear.regression-3.pdf}

\begin{Shaded}
\begin{Highlighting}[]
\FunctionTok{plot}\NormalTok{(temperature.depth.regreesion, }\DecValTok{4}\NormalTok{)}
\end{Highlighting}
\end{Shaded}

\includegraphics{DwitiBagadia_A06_GLMs_files/figure-latex/linear.regression-4.pdf}

\begin{enumerate}
\def\labelenumi{\arabic{enumi}.}
\setcounter{enumi}{7}
\tightlist
\item
  Interpret your model results in words. Include how much of the
  variability in temperature is explained by changes in depth, the
  degrees of freedom on which this finding is based, and the statistical
  significance of the result. Also mention how much temperature is
  predicted to change for every 1m change in depth.
\end{enumerate}

\begin{quote}
Answer: There is a significant negative correlation (p value \textless{}
2.2e-16) between temperature and depth with around 9726 degrees of
freedom(df). This model helps to explain 73.87\% of variance in
temperature.
\end{quote}

\begin{center}\rule{0.5\linewidth}{0.5pt}\end{center}

\hypertarget{multiple-regression}{%
\subsection{Multiple regression}\label{multiple-regression}}

Let's tackle a similar question from a different approach. Here, we want
to explore what might the best set of predictors for lake temperature in
July across the monitoring period at the North Temperate Lakes LTER.

\begin{enumerate}
\def\labelenumi{\arabic{enumi}.}
\setcounter{enumi}{8}
\item
  Run an AIC to determine what set of explanatory variables (year4,
  daynum, depth) is best suited to predict temperature.
\item
  Run a multiple regression on the recommended set of variables.
\end{enumerate}

\begin{Shaded}
\begin{Highlighting}[]
\CommentTok{\#9 running AIC to determine best suited set of variables to predict temeprature }
\NormalTok{NTL\_LTER.aic }\OtherTok{\textless{}{-}} \FunctionTok{lm}\NormalTok{(}\AttributeTok{data =}\NormalTok{ NTL\_LTER.selected, temperature\_C }\SpecialCharTok{\textasciitilde{}}\NormalTok{ year4 }\SpecialCharTok{+}\NormalTok{ daynum }\SpecialCharTok{+}\NormalTok{ depth)}
\FunctionTok{step}\NormalTok{(NTL\_LTER.aic)}
\end{Highlighting}
\end{Shaded}

\begin{verbatim}
## Start:  AIC=966.18
## temperature_C ~ year4 + daynum + depth
## 
##          Df Sum of Sq     RSS     AIC
## - year4   1      21.7  4505.8  966.08
## <none>                 4484.0  966.18
## - daynum  1     638.3  5122.3 1016.62
## - depth   1   15263.4 19747.5 1548.29
## 
## Step:  AIC=966.08
## temperature_C ~ daynum + depth
## 
##          Df Sum of Sq     RSS     AIC
## <none>                 4505.8  966.08
## - daynum  1       717  5222.8 1022.27
## - depth   1     15242 19747.5 1546.29
\end{verbatim}

\begin{verbatim}
## 
## Call:
## lm(formula = temperature_C ~ daynum + depth, data = NTL_LTER.selected)
## 
## Coefficients:
## (Intercept)       daynum        depth  
##    11.19860      0.05466     -1.91767
\end{verbatim}

\begin{Shaded}
\begin{Highlighting}[]
\CommentTok{\#10 running multiple regression on the recommended set of variables}
\NormalTok{temperature.best }\OtherTok{\textless{}{-}} \FunctionTok{lm}\NormalTok{(}\AttributeTok{data =}\NormalTok{ NTL\_LTER.selected, temperature\_C }\SpecialCharTok{\textasciitilde{}}\NormalTok{ year4 }\SpecialCharTok{+}\NormalTok{ daynum }\SpecialCharTok{+}\NormalTok{ depth)}
\FunctionTok{summary}\NormalTok{(temperature.best)}
\end{Highlighting}
\end{Shaded}

\begin{verbatim}
## 
## Call:
## lm(formula = temperature_C ~ year4 + daynum + depth, data = NTL_LTER.selected)
## 
## Residuals:
##     Min      1Q  Median      3Q     Max 
## -8.3086 -2.7609 -0.3194  2.5294  8.1964 
## 
## Coefficients:
##              Estimate Std. Error t value Pr(>|t|)    
## (Intercept) 123.85217   81.96700   1.511    0.132    
## year4        -0.05591    0.04067  -1.375    0.170    
## daynum        0.05268    0.00707   7.451 6.02e-13 ***
## depth        -1.92045    0.05271 -36.435  < 2e-16 ***
## ---
## Signif. codes:  0 '***' 0.001 '**' 0.01 '*' 0.05 '.' 0.1 ' ' 1
## 
## Residual standard error: 3.391 on 390 degrees of freedom
## Multiple R-squared:  0.7802, Adjusted R-squared:  0.7785 
## F-statistic: 461.5 on 3 and 390 DF,  p-value: < 2.2e-16
\end{verbatim}

\begin{enumerate}
\def\labelenumi{\arabic{enumi}.}
\setcounter{enumi}{10}
\tightlist
\item
  What is the final set of explanatory variables that the AIC method
  suggests we use to predict temperature in our multiple regression? How
  much of the observed variance does this model explain? Is this an
  improvement over the model using only depth as the explanatory
  variable?
\end{enumerate}

\begin{quote}
Answer: The final set of explanatory variables that the AIC method
suggests we use to predict temperature in our multiple regression are
year, day number and depth. This model explains 74\% of the total
observed variance. This is a slight improvement from the previous model
of just depth as the singular explanatory variable, increasing the
R-squared by .01.
\end{quote}

\begin{center}\rule{0.5\linewidth}{0.5pt}\end{center}

\hypertarget{analysis-of-variance}{%
\subsection{Analysis of Variance}\label{analysis-of-variance}}

\begin{enumerate}
\def\labelenumi{\arabic{enumi}.}
\setcounter{enumi}{11}
\tightlist
\item
  Now we want to see whether the different lakes have, on average,
  different temperatures in the month of July. Run an ANOVA test to
  complete this analysis. (No need to test assumptions of normality or
  similar variances.) Create two sets of models: one expressed as an
  ANOVA models and another expressed as a linear model (as done in our
  lessons).
\end{enumerate}

\begin{Shaded}
\begin{Highlighting}[]
\CommentTok{\#12}
\FunctionTok{library}\NormalTok{(htmltools)}
\end{Highlighting}
\end{Shaded}

\begin{verbatim}
## Warning: package 'htmltools' was built under R version 4.2.2
\end{verbatim}

\begin{Shaded}
\begin{Highlighting}[]
\NormalTok{NTL\_LTER.ANOVA }\OtherTok{\textless{}{-}} \FunctionTok{aov}\NormalTok{(}\AttributeTok{data =}\NormalTok{ NTL\_LTER.selected, temperature\_C }\SpecialCharTok{\textasciitilde{}}\NormalTok{ lakename)}
\FunctionTok{summary}\NormalTok{(NTL\_LTER.ANOVA)}
\end{Highlighting}
\end{Shaded}

\begin{verbatim}
##              Df Sum Sq Mean Sq F value Pr(>F)
## lakename      4    228   56.96   1.098  0.357
## Residuals   389  20176   51.87
\end{verbatim}

\begin{Shaded}
\begin{Highlighting}[]
\CommentTok{\#rejecting null hypothesis}

\NormalTok{NTL\_LTER.linreg }\OtherTok{\textless{}{-}} \FunctionTok{lm}\NormalTok{(}\AttributeTok{data =}\NormalTok{ NTL\_LTER.selected, temperature\_C }\SpecialCharTok{\textasciitilde{}}\NormalTok{ lakename)}
\FunctionTok{summary}\NormalTok{(NTL\_LTER.linreg)}
\end{Highlighting}
\end{Shaded}

\begin{verbatim}
## 
## Call:
## lm(formula = temperature_C ~ lakename, data = NTL_LTER.selected)
## 
## Residuals:
##    Min     1Q Median     3Q    Max 
## -8.331 -6.756 -2.550  7.338 14.536 
## 
## Coefficients:
##                      Estimate Std. Error t value Pr(>|t|)    
## (Intercept)           10.5556     1.6975   6.218  1.3e-09 ***
## lakenamePaul Lake      1.9008     1.7856   1.065    0.288    
## lakenamePeter Lake     2.0088     1.7904   1.122    0.263    
## lakenameTuesday Lake  -0.4389     2.4006  -0.183    0.855    
## lakenameWard Lake      3.3755     2.1610   1.562    0.119    
## ---
## Signif. codes:  0 '***' 0.001 '**' 0.01 '*' 0.05 '.' 0.1 ' ' 1
## 
## Residual standard error: 7.202 on 389 degrees of freedom
## Multiple R-squared:  0.01117,    Adjusted R-squared:  0.0009981 
## F-statistic: 1.098 on 4 and 389 DF,  p-value: 0.3571
\end{verbatim}

\begin{enumerate}
\def\labelenumi{\arabic{enumi}.}
\setcounter{enumi}{12}
\tightlist
\item
  Is there a significant difference in mean temperature among the lakes?
  Report your findings.
\end{enumerate}

\begin{quote}
Answer: There is a significant difference in mean temperatures among the
lakes. This model explains about 4\% of the total variance in
temperature.
\end{quote}

\begin{enumerate}
\def\labelenumi{\arabic{enumi}.}
\setcounter{enumi}{13}
\tightlist
\item
  Create a graph that depicts temperature by depth, with a separate
  color for each lake. Add a geom\_smooth (method = ``lm'', se = FALSE)
  for each lake. Make your points 50 \% transparent. Adjust your y axis
  limits to go from 0 to 35 degrees. Clean up your graph to make it
  pretty.
\end{enumerate}

\begin{Shaded}
\begin{Highlighting}[]
\CommentTok{\#14. scatter plots}
\FunctionTok{unique}\NormalTok{(NTL\_LTER.selected}\SpecialCharTok{$}\NormalTok{lakename)}
\end{Highlighting}
\end{Shaded}

\begin{verbatim}
## [1] Peter Lake     Paul Lake      East Long Lake Ward Lake      Tuesday Lake  
## 9 Levels: Central Long Lake Crampton Lake East Long Lake ... West Long Lake
\end{verbatim}

\begin{Shaded}
\begin{Highlighting}[]
\NormalTok{temperature.depth}\FloatTok{.2} \OtherTok{\textless{}{-}} 
  \FunctionTok{ggplot}\NormalTok{(NTL\_LTER.selected) }\SpecialCharTok{+} 
  \FunctionTok{geom\_point}\NormalTok{(}\FunctionTok{aes}\NormalTok{(}\AttributeTok{x =}\NormalTok{ depth, }\AttributeTok{y =}\NormalTok{ temperature\_C, }\AttributeTok{color =}\NormalTok{ lakename), }\AttributeTok{alpha =} \FloatTok{0.5}\NormalTok{) }\SpecialCharTok{+}
  \FunctionTok{geom\_smooth}\NormalTok{(}\FunctionTok{aes}\NormalTok{(}\AttributeTok{x =}\NormalTok{ depth, }\AttributeTok{y =}\NormalTok{ temperature\_C, }\AttributeTok{color =}\NormalTok{ lakename), }\AttributeTok{method =} \StringTok{"lm"}\NormalTok{, }\AttributeTok{se =} \ConstantTok{FALSE}\NormalTok{) }\SpecialCharTok{+}
  \FunctionTok{ylim}\NormalTok{(}\DecValTok{0}\NormalTok{, }\DecValTok{35}\NormalTok{) }\SpecialCharTok{+} 
  \FunctionTok{labs}\NormalTok{(}\AttributeTok{x =} \StringTok{"Depth(Mts.)"}\NormalTok{, }\AttributeTok{y =} \StringTok{"Temperature(C)"}\NormalTok{) }\SpecialCharTok{+} 
\NormalTok{  tweetheme}
\FunctionTok{print}\NormalTok{(temperature.depth}\FloatTok{.2}\NormalTok{)}
\end{Highlighting}
\end{Shaded}

\begin{verbatim}
## `geom_smooth()` using formula = 'y ~ x'
\end{verbatim}

\begin{verbatim}
## Warning: Removed 13 rows containing missing values (`geom_smooth()`).
\end{verbatim}

\includegraphics{DwitiBagadia_A06_GLMs_files/figure-latex/scatterplot.2-1.pdf}

\begin{enumerate}
\def\labelenumi{\arabic{enumi}.}
\setcounter{enumi}{14}
\tightlist
\item
  Use the Tukey's HSD test to determine which lakes have different
  means.
\end{enumerate}

\begin{Shaded}
\begin{Highlighting}[]
\CommentTok{\#15}
\FunctionTok{TukeyHSD}\NormalTok{(NTL\_LTER.ANOVA)}
\end{Highlighting}
\end{Shaded}

\begin{verbatim}
##   Tukey multiple comparisons of means
##     95% family-wise confidence level
## 
## Fit: aov(formula = temperature_C ~ lakename, data = NTL_LTER.selected)
## 
## $lakename
##                                   diff       lwr      upr     p adj
## Paul Lake-East Long Lake     1.9007758 -2.992848 6.794400 0.8245987
## Peter Lake-East Long Lake    2.0088194 -2.898035 6.915674 0.7948864
## Tuesday Lake-East Long Lake -0.4388889 -7.018014 6.140236 0.9997496
## Ward Lake-East Long Lake     3.3754789 -2.546994 9.297952 0.5227817
## Peter Lake-Paul Lake         0.1080436 -2.069085 2.285172 0.9999228
## Tuesday Lake-Paul Lake      -2.3396647 -7.233289 2.553959 0.6849800
## Ward Lake-Paul Lake          1.4747031 -2.492456 5.441862 0.8466692
## Tuesday Lake-Peter Lake     -2.4477083 -7.354562 2.459146 0.6491273
## Ward Lake-Peter Lake         1.3666595 -2.616808 5.350127 0.8810112
## Ward Lake-Tuesday Lake       3.8143678 -2.108105 9.736840 0.3955788
\end{verbatim}

\begin{Shaded}
\begin{Highlighting}[]
\NormalTok{NTL\_LTER.group }\OtherTok{\textless{}{-}} \FunctionTok{HSD.test}\NormalTok{(NTL\_LTER.ANOVA, }\StringTok{"lakename"}\NormalTok{, }\AttributeTok{group =} \ConstantTok{TRUE}\NormalTok{)}
\NormalTok{NTL\_LTER.group}
\end{Highlighting}
\end{Shaded}

\begin{verbatim}
## $statistics
##    MSerror  Df     Mean       CV
##   51.86594 389 12.41503 58.00875
## 
## $parameters
##    test   name.t ntr StudentizedRange alpha
##   Tukey lakename   5         3.875817  0.05
## 
## $means
##                temperature_C      std   r Min  Max   Q25   Q50   Q75
## East Long Lake      10.55556 5.428923  18 4.7 17.4 5.525  8.35 17.15
## Paul Lake           12.45633 6.832892 169 4.8 24.2 5.800 10.20 19.70
## Peter Lake          12.56438 7.767335 160 4.3 27.1 5.000  9.35 20.70
## Tuesday Lake        10.11667 6.638502  18 4.3 21.0 4.625  6.40 16.30
## Ward Lake           13.93103 7.293006  29 5.6 25.5 7.100 12.50 21.30
## 
## $comparison
## NULL
## 
## $groups
##                temperature_C groups
## Ward Lake           13.93103      a
## Peter Lake          12.56438      a
## Paul Lake           12.45633      a
## East Long Lake      10.55556      a
## Tuesday Lake        10.11667      a
## 
## attr(,"class")
## [1] "group"
\end{verbatim}

16.From the findings above, which lakes have the same mean temperature,
statistically speaking, as Peter Lake? Does any lake have a mean
temperature that is statistically distinct from all the other lakes?

\begin{quote}
Answer: Statistically speaking, Paul lake and Ward Lake have the same
mean temperature as Peter Lake. Central Long Lake has a distinct mean
temperature from most of the other lakes except from Crampton, hence, no
lake has a mean temperature that is statistically distinct from all the
other lakes.
\end{quote}

\begin{enumerate}
\def\labelenumi{\arabic{enumi}.}
\setcounter{enumi}{16}
\tightlist
\item
  If we were just looking at Peter Lake and Paul Lake. What's another
  test we might explore to see whether they have distinct mean
  temperatures?
\end{enumerate}

\begin{quote}
Answer: We could perform a two-way t test.
\end{quote}

\begin{enumerate}
\def\labelenumi{\arabic{enumi}.}
\setcounter{enumi}{17}
\tightlist
\item
  Wrangle the July data to include only records for Crampton Lake and
  Ward Lake. Run the two-sample T-test on these data to determine
  whether their July temperature are same or different. What does the
  test say? Are the mean temperatures for the lakes equal? Does that
  match you answer for part 16?
\end{enumerate}

\begin{Shaded}
\begin{Highlighting}[]
\NormalTok{NTL\_LTER.ward.crampton }\OtherTok{\textless{}{-}}\NormalTok{ NTL\_LTER.selected }\SpecialCharTok{\%\textgreater{}\%}
  \FunctionTok{filter}\NormalTok{(lakename}\SpecialCharTok{\%in\%} \FunctionTok{c}\NormalTok{(}\StringTok{"Crampton Lake"}\NormalTok{, }\StringTok{"Ward Lake"}\NormalTok{))}
\end{Highlighting}
\end{Shaded}

\begin{quote}
Answer:
\end{quote}

\end{document}
